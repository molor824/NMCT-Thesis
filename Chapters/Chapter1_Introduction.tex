\chapter{Удиртгал}
\label{ch:introduction}

\section{Үндэслэл, ач холбогдол}

Валютын зах зээл (Foreign Exchange буюу Forex) нь дэлхийн хамгийн том санхүүгийн зах зээл бөгөөд өдөр тутмын арилжааны хэмжээ 7.5 их наяд ам.долларт хүрдэг \cite{BIS2022}. Энэхүү зах зээл нь 24 цагийн турш ажилладаг, олон улсын эдийн засаг, геополитик нөхцөл байдал, төв банкуудын бодлого зэрэг олон тооны хүчин зүйлээс хамаардаг нь үнийн хөдөлгөөнийг таамаглахад ихээхэн хүндрэлийг учруулдаг.

Уламжлалт техник дүн шинжилгээний аргууд нь хүний туршлага, мэдлэгт тулгуурладаг тул хэлбэлзэл ихтэй, өндөр савалгаатай арилжаанд тэдгээрийн үр ашиг хязгаарлагдмал байдаг. Мөн хүний шийдвэр нь сэтгэл хөдлөл, субъектив үнэлгээнээс шалтгаалан алдаа гарах магадлал өндөр байдаг.

Сүүлийн жилүүдэд машин сургалт (Machine Learning) болон хиймэл оюун ухааны технологиуд санхүүгийн салбарт өргөн хэрэглэгдэж эхэлсэн. Gradient Boosted Decision Trees (GBDT), нейрон сүлжээ зэрэг орчин үеийн алгоритмууд нь том хэмжээний өгөгдөл боловсруулж, нарийн төвөгтэй хэв маягуудыг илрүүлэх чадвартай болсон \cite{chen2016xgboost, ke2017lightgbm}. Гэвч overfitting-ийн асуудал, практик хэрэглээний хүртээмжийн дутагдал, дан загварын хязгаарлалт зэрэг сорилтууд одоо ч байсаар байна.

Энэхүү судалгаа нь ансамбль загваруудын хослол, walk-forward validation аргачлал, олон хугацааны интервалаас техник индикаторын инженерчлэл (Feature Engineering) ашигласнаар Forex таамаглалын судалгаанд хувь нэмэр оруулж, илүү найдвартай, тогтвортой систем бүтээх боломжийг судална. Түүнчлэн хувь хүн, жижиг арилжаачдад хүртээмжтэй, хэрэглэхэд хялбар, үнэ төлбөргүй систем нь Forex зах зээлд оролцох боломжийг нээж өгч, Монгол Улсын FinTech салбарын хөгжилд хувь нэмрээ оруулах боломжтой.

\section{Зорилго, зорилт}

\subsection{Судалгааны зорилго}

Энэхүү судалгааны ажлын гол зорилго нь \textbf{машин сургалтын ансамбль аргуудыг ашиглан EUR/USD валютын хослолын үнийн чиг хандлагыг таамаглах, автомат арилжааны дохио үүсгэх найдвартай систем болон хэрэглэгчдэд хүртээмжтэй гар утасны аппликейшн хөгжүүлэх} явдал юм.

\subsection{Судалгааны зорилтууд}

Дээрх зорилгод хүрэхийн тулд дараах зорилтуудыг дэвшүүлсэн:

\begin{enumerate}
    \item Forex валютын зах зээл болон машин сургалтын таамаглалын аргуудын судалгааны үндэслэлийг бий болгох.
    \item Ансамбль загваруудын хослол, олон хугацааны интервалын дүн шинжилгээ, walk-forward validation аргачлал бүхий найдвартай арилжааны дохионы систем хөгжүүлэх.
    \item Бодит зах зээлийн нөхцөлд системийн практик үр дүн, эрсдэл, ашигт ажиллагааг шалгаж үнэлэх.
    \item Хэрэглэгчдэд хүртээмжтэй, ойлгомжтой интерфейс бүхий гар утасны аппликейшн бүтээж, ML дохио болон зах зээлийн мэдээллийг нэгтгэх.
\end{enumerate}
