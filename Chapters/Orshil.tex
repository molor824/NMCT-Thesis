\addtotoc{Оршил}
\chapter*{Оршил}

\section*{Судалгааны сэдвийн тодорхойлолт}

Валютын зах зээл (Foreign Exchange буюу Forex) нь дэлхийн хамгийн том, хөрвөх чадвар өндөртэй санхүүгийн зах зээл бөгөөд өдөр тутмын арилжааны хэмжээ нь 7.5 их наяд ам.долларт хүрдэг. Энэхүү зах зээл нь 24 цагийн турш ажилладаг, олон улсын эдийн засаг, геополитик нөхцөл байдал, төв банкуудын бодлого зэрэг олон тооны хүчин зүйлээс хамаардаг нь үнийн хөдөлгөөнийг таамаглахад ихээхэн хүндрэлийг учруулдаг.

Уламжлалт техник дүн шинжилгээний аргууд болон суурь дүн шинжилгээ нь хүний туршлага, мэдлэгт ихээхэн найддаг боловч хэлбэлзэл ихтэй, өндөр давтамжтай арилжаанд тэдгээрийн үр ашиг хязгаарлагдмал байдаг. Мөн хүний шийдвэр нь сэтгэл хөдлөл, субъектив үнэлгээнээс нөлөөлөгдөж болзошгүй нь алдаа гаргах магадлалыг нэмэгдүүлдэг.

\section*{Судалгааны шаардлага ба актуальность}

Сүүлийн арваад жилд машин сургалт (Machine Learning) болон хиймэл оюун ухааны технологиуд санхүүгийн салбарт өргөн хэрэглэгдэж эхэлсэн. Gradient Boosted Decision Trees (GBDT), нейрон сүлжээ зэрэг орчин үеийн алгоритмууд нь том хэмжээний өгөгдөл боловсруулж, нарийн төвөгтэй хэв маягуудыг илрүүлэх чадвартай болсон. Гэвч дараах асуудлууд одоо ч байсаар байна:

\begin{itemize}
    \item \textbf{Overfitting:} Олон загвар нь сургалтын өгөгдөл дээр сайн ажилладаг боловч бодит зах зээл дээр муу үр дүн үзүүлдэг.
    \item \textbf{Хүртээмж:} Ихэнх судалгаанууд академик түвшинд хийгдсэн, практик хэрэглэгчдэд хүртээмжтэй бус.
    \item \textbf{Дан загварын хязгаарлалт:} Нэг загвар ашигладаг систем нь тогтвортой бус, эрсдэл өндөртэй.
    \item \textbf{Бодит цагийн мэдээлэл дутагдал:} Загвар дангаараа хүрэлцэхгүй, хэрэглэгчдэд ойлгомжтой интерфейс, шинэ мэдээлэл хэрэгтэй.
\end{itemize}

Эдгээр асуудлыг шийдвэрлэхийн тулд ансамбль загварууд, walk-forward validation аргачлал, мөн гар утасны аппликейшн хөгжүүлэх нь зайлшгүй шаардлагатай болсон.

\section*{Судалгааны зорилго}

Энэхүү судалгааны ажлын гол зорилго нь \textbf{машин сургалтын ансамбль аргуудыг ашиглан EUR/USD валютын хослолын үнийн чиг хандлагыг таамаглах, автомат арилжааны дохио үүсгэх найдвартай систем болон гар утасны аппликейшн хөгжүүлэх} явдал юм.

\section*{Судалгааны зорилтууд}

Дээрх зорилгод хүрэхийн тулд дараах зорилтуудыг дэвшүүлсэн:

\begin{enumerate}
    \item Forex валютын зах зээлийн онцлог, техник дүн шинжилгээний үндсэн ойлголтууд, машин сургалтын аргуудыг судлах.
    \item LightGBM, XGBoost, CatBoost гэсэн гурван GBDT загварын ансамбль системийг бүтээх.
    \item 2015--2024 оны EUR/USD түүхэн өгөгдөл ашиглан загваруудыг сургах, баталгаажуулах, тестлэх.
    \item Walk-forward validation аргачлалыг хэрэгжүүлж overfitting-ийн эрсдэлийг бууруулах.
    \item 6 хугацааны интервалаас (M1, M5, M15, M30, H1, H4) 48 техник индикаторуудыг тооцоолох, олон хугацааны дүн шинжилгээ хийх.
    \item MetaTrader 5 платформ дээр Expert Advisor ашиглан 2025 оны backtest хийх.
    \item React Native ашиглан ``Predictrix'' нэртэй гар утасны аппликейшн хөгжүүлж, бодит цагийн мэдээлэл, ML дохио, эдийн засгийн мэдээ зэрэг функцүүдийг нэгтгэх.
    \item Системийн гүйцэтгэл, найдвартай байдлыг үнэлэх, практик хэрэглээний чиглэлийг тодорхойлох.
\end{enumerate}

\section*{Судалгааны ач холбогдол}

Энэхүү судалгааны ажлын ач холбогдол нь:

\begin{itemize}
    \item \textbf{Шинжлэх ухааны ач холбогдол:} Ансамбль загваруудын хослол, walk-forward validation аргачлалын хэрэглээ, олон хугацааны интервалаас техник индикатор гаргаж ашигласан нь Forex таамаглалын судалгаанд хувь нэмэр оруулна.
    \item \textbf{Практик ач холбогдол:} Хувь хүн, жижиг арилжаачдад хүртээмжтэй, хэрэглэхэд хялбар, үнэ төлбөргүй систем нь Forex зах зээлд оролцох боломжийг нээж өгнө.
    \item \textbf{Технологийн ач холбогдол:} Орчин үеийн Machine Learning, мобайл хөгжүүлэлт, backend системүүдийг нэгтгэсэн end-to-end шийдэл бий болгосон нь Монгол Улсын FinTech салбарын хөгжилд хувь нэмэр оруулна.
\end{itemize}

Судалгааны ажил нь загварын сургалтаас эхлээд мобайл аппликейшн хүртэлх бүрэн end-to-end системийг хамарсан цогц инженерийн ажил юм.
