\addtotoc{Товчлол}
\chapter*{Товчлол}

\vspace{5mm}
\begin{center}
\textbf{\large \titlename}
\end{center}

\begin{center}
\authorsname\\
\universityname, \departmentname\\
\textit{Цахим шуудан: munkhsuld@nmct.edu.mn}
\end{center}

\vspace{3mm}
\noindent\textbf{Хураангуй:} Энэхүү судалгааны ажлаар машин сургалтын ансамбль аргуудыг ашиглан EUR/USD валютын хослолын үнийн чиг хандлагыг таамаглах систем хөгжүүлж, +41.61\% өгөөж, Sharpe Ratio 9.64, Max Drawdown 3.93\% хүрсэн үр дүнтэй байсан.

\vspace{3mm}
\noindent\textit{Түлхүүр үг: машин сургалт, ансамбль загвар, валютын зах зээл}

\section*{1. Удиртгал}

Валютын зах зээл (Forex) нь дэлхийн хамгийн том санхүүгийн зах зээл бөгөөд өдөр тутмын арилжааны хэмжээ 7.5 их наяд ам.долларт хүрдэг. Энэхүү зах зээлийн хэт хэлбэлзэлтэй байдал нь уламжлалт аргуудаар үнийн хөдөлгөөнийг таамаглахад хүндрэл учруулдаг. Сүүлийн жилүүдэд машин сургалтын технологиуд энэ салбарт өргөн хэрэглэгдэх болсон ч ихэнх систем нь дангаар ажилладаг загварт тулгуурладаг бөгөөд overfitting-ийн эрсдэл өндөртэй. Энэхүү судалгааны зорилго нь олон загварын ансамбль аргыг ашиглан найдвартай арилжааны дохио үүсгэх, мобайл аппликейшнаар хэрэглэгчдэд хүргэх бүрэн системийг хөгжүүлэх юм.

\section*{2. Судалгааны арга зүй}

Судалгааны ажилд LightGBM, XGBoost, CatBoost гэсэн гурван Gradient Boosted Decision Trees (GBDT) загварын ансамблийг ашигласан. MetaTrader 5-аас EUR/USD валютын хослолын 2015--2024 оны OHLCV өгөгдлийг 6 хугацааны интервалаар (M1, M5, M15, M30, H1, H4) татан авч, интервал бүрээс 8 техник индикатор тооцоолж нийт 48 техник индикатор бүхий өгөгдлийн бүтэц үүсгэсэн.

Walk-forward validation аргачлалаар өгөгдлийг хуваасан: сургалт (2015--2022, 80\%), баталгаажуулалт (2023, 10\%), тест (2024, 10\%). Загварын таамгийг Logistic Regression calibrator-аар нэгтгэж, confidence $\geq$ 90\%, ATR $\geq$ 4.0 пипс шүүлтүүр хэрэглэн чанартай дохио үүсгэсэн.

Flask REST API сервер дээр ажилладаг backend систем, React Native технологиор мобайл аппликейшн хөгжүүлсэн. MetaTrader 5 Strategy Tester дээр бодит зах зээлийн нөхцөлд backtest хийсэн.

\section*{3. Судалгааны үр дүн}

2025 оны 01--10 сарын backtest-ийн гол үр дүн:
\begin{itemize}
    \item Өгөөж: +41.61\% (\$10,000 $\to$ \$14,161.20)
    \item Profit Factor: 2.46 (нийт ашиг нь алдагдлаас 2.46 дахин их)
    \item Sharpe Ratio: 9.64 (эрсдэлд тохируулсан өгөөж маш өндөр)
    \item Max Drawdown: 3.93\% (маш бага уналт)
    \item Нийт 45 арилжаа, 44.44\% win rate, дундаж ашиг \$351 vs дундаж алдагдал \$114
\end{itemize}

7 давталтат хөгжүүлэлтийн үе шатаар загварыг 75-аас 48 индикатор, 9-өөс 3 загвар болгож хялбаршуулсан ч бүх гүйцэтгэлийн хэмжүүрүүд сайжирсан. Тест дээрх нарийвчлал 87.4\%, өндөр итгэлцэлтэй дохионы нарийвчлал 95\%+ байсан нь overfitting байхгүй болохыг баталж байна.

\section*{4. Дүгнэлт}

Машин сургалтын ансамбль загвар нь Forex зах зээл дээр ашигтай арилжааны дохио үүсгэх чадвартай болохыг энэхүү судалгаа бодитоор батлав. Систем нь мэргэжлийн хөрөнгө оруулалтын сангийн түвшний гүйцэтгэл үзүүлсэн. Гэвч ганц валютын хослолд сургагдсан байдал, зах зээлийн горим өөрчлөгдөх эрсдэл зэрэг хязгаарлалтууд байна. Цаашид олон хослолд өргөтгөх, гүн сургалтын загвар нэмэх, NLP-ээр мэдээний шинжилгээ хийх чиглэлээр хөгжүүлэх боломжтой.

Судалгааны ажлын цаашдын төлөвлөгөө болон хэтийн төлөвийг хэлэлцүүлэг байдлаар бичиж оруулна.

\section*{5. Ном зүй}

Судалгаанд ашигласан гол ном, сурах бичиг, эрдэм шинжилгээний өгүүллүүдээс:
\begin{enumerate}
    \item Chen, T. and Guestrin, C. (2016). ``XGBoost: A Scalable Tree Boosting System.'' \textit{ACM SIGKDD}.
    \item Ke, G. et al. (2017). ``LightGBM: A Highly Efficient Gradient Boosting Decision Tree.'' \textit{NeurIPS}.
    \item Prokhorenkova, L. et al. (2018). ``CatBoost: Unbiased Boosting with Categorical Features.'' \textit{NeurIPS}.
    \item Gu, S. et al. (2020). ``Empirical Asset Pricing via Machine Learning.'' \textit{The Review of Financial Studies}.
    \item Pardo, R. (2008). \textit{The Evaluation and Optimization of Trading Strategies}. John Wiley \& Sons.
\end{enumerate}
